% outer theme
\useoutertheme[footline=authortitle,subsection=false]{miniframes}

% inner theme
\useinnertheme{circles}

% color themes
\usecolortheme{whale}
\usecolortheme{orchid}

% font themes
\usefonttheme{structurebold}
\usefonttheme{serif}

% move frame title to the left and to the top, set color
\setbeamertemplate{frametitle}[default][left,leftskip=-1mm]
\addtobeamertemplate{frametitle}{\vspace*{-1mm}}{\vspace*{1mm}}
\setbeamercolor{frametitle}{fg=mittelblau,bg=white}

% set text margin
\setbeamersize{text margin left=4mm,text margin right=3mm}

% basic color for all structuring elements (e.g., bullet points)
\setbeamercolor{structure}{fg=mittelblau}

% don't use balls for enumerate items, use plain black font
\setbeamertemplate{enumerate item}[default]
\setbeamertemplate{enumerate subitem}[default]
\setbeamertemplate{enumerate subsubitem}[default]
\setbeamercolor{item}{fg=black}
\setbeamerfont{item}{series=\mdseries}

% font and color of frame number
% \sffamily\fontsize{3mm}{3mm}\selectfont\bfseries
\setbeamerfont{footline}{
  family=\sffamily,
  series=\bfseries,
  size={\fontsize{2.5mm}{2.5mm}},
}
\setbeamercolor{footline}{fg=white}

% block title
\setbeamercolor{block title}{bg=mittelblau}

% bibliography colors
% (only emphasize the title with the mittelblau color,
% all other text should be black)
\setbeamercolor{bibliography entry author}{use=normal text,fg=normal text.fg}
\setbeamercolor{bibliography entry title}{use=structure,fg=structure.fg}
\setbeamercolor{bibliography entry location}{use=normal text,fg=normal text.fg}
\setbeamercolor{bibliography entry note}{use=normal text,fg=normal text.fg}

% transitions
%\renewcommand*{\beamer@framenotesbegin}{%
%  \transfade[duration=0.2]%
%  \transfade<1>[duration=0.5]%
%}

% make covered text transparent (opacity in percent)
\newcommand*{\uncoveropacityint}{8}
\setbeamercovered{transparent=\uncoveropacityint}

% scale for including full HD graphics (cover/footer)
% = 96dpi / (1920px / (160mm / 25.4mm/in))
\newcommand*{\pngscale}{0.3149606}

% title frame
\defbeamertemplate*{title page}{customized}[1][]{
  \begin{overlay}
    \node at (0,0) {%
      \includegraphics[scale=\pngscale]{cover}%
    };
  \end{overlay}
}
\newcommand*{\titleframe}{%
  \section*{}
  {%
    \setbeamertemplate{headline}{}
    \setbeamertemplate{footline}{}
    \begin{frame}[noframenumbering]
      \titlepage
    \end{frame}
  }%
}

% header (empty)
\defbeamertemplate*{headline}{myminiframes theme}{}

% footer (bar with frame number)
\setbeamertemplate{footline}{%
  \raisebox{-0.5\height}{\includegraphics[scale=\pngscale]{footerBar}}%
  \raisebox{-0.5\height}{\hspace{-1mm}\llap{\insertframenumber}}%
}

% hide navigation symbols
\beamertemplatenavigationsymbolsempty

% don't replace font-related stuff
% (mathdesign is not supported by beamerbasefont),
% otherwise the l --> \ell replacement in math mode doesn't work
\beamer@suppressreplacementstrue

% absolute positioning of contents
\newenvironment*{overlay}[1][]{%
  \begin{tikzpicture}[
    remember picture,
    overlay,
    x=1mm,
    y=-1mm,
    shift={(current page.north west)},
    every node/.style={anchor=north west,align=flush left,inner sep=0mm},
    tight background,
    background grid/.style={
      draw,
      black!50,
      step=10mm,
      shift={(current page.north west)},
    },
    #1
  ]%
}{%
  \end{tikzpicture}%
}

% define \hvisible and \vvisible: similar to \visible,
% except that they only reserve horizontal or vertical space
% if not visible on current slide
% (\visible reserves both horizontal and vertical space)
\newcommand<>{\hvisible}[1]{\alt#2{#1}{\hphantom{#1}}}
\newcommand<>{\vvisible}[1]{\alt#2{#1}{\vphantom{#1}}}

% define \guncover: similar to \uncover, except that it works
% for graphics (\uncover has no effect on those)
\pgfmathsetmacro{\uncoveropacity}{\uncoveropacityint/100}
\pgfmathsetmacro{\uncoveralpha}{1-\uncoveropacity}
\newcommand<>{\guncover}[1]{%
  \alt#2{#1}{{\transparent{\uncoveropacity}#1}}%
}

% remove footnote rule
\renewcommand*{\footnoterule}{}

% remove indentation in footnotes
\defbeamertemplate*{footnote}{customized}{%
  \noindent%
  \insertfootnotemark\insertfootnotetext\par%
}
