\section{Introduction}



\begin{frame}<handout:0>[noframenumbering]
  \begin{overlay}
    \sectionCircle
  \end{overlay}
\end{frame}



\subsection{Inverse Problems}

\begin{frame}{\insertsubsection}
  \begin{overlay}
    \node[anchor=north,align=center,text width=140mm] at (80,16) {%
      Inverse problems help to simulate reality with
      cheap(er) models.\\[1em]
      But how can we efficiently solve inverse problems or other
      optimization problems in higher dimensions?%
    };
    
    \begin{scope}[
      tornpaper/.style={
        decorate,
        decoration={random steps,segment length=1mm,amplitude=0.6mm}
      },
      xshift=10,yshift=-125,
    ]
      \newcommand*{\tph}{28}
      \pgfmathsetseed{342}
      \begin{scope}[xshift=3,yshift=-3]
        \fill[black!30] (0,0) -- (50,0) -- (50,\tph)
            decorate[tornpaper]{-- (0,\tph)} -- cycle;
      \end{scope}
      \pgfmathsetseed{342}
      \draw[black,fill=yellow!20] (0,0) -- (50,0) -- (50,\tph)
          decorate[tornpaper]{-- (0,\tph)} -- cycle;
      \pgfmathsetseed{342}
      \clip (0,0) -- (50,0) -- (50,\tph)
          decorate[tornpaper]{-- (0,\tph)} -- cycle;
      \node[text width=50mm,anchor=north west,inner sep=1.5mm] at (0,0) {%
        \textbf{Raspberry Cheesecake}\\
        \emph{Ingredients:}\\
        200g flour\\
        50g ground almonds\\
        80g powerded sugar\\
        5 eggs%
      };
      \fill[brown!70,rotate=70] (-14,9) ellipse (5 and 3);
    \end{scope}
    
    \node[text width=95mm] at (60,50) {%
      ``It is easy to make a cake from a recipe; but can we write\\
      \hspace{0.25em}down the recipe if we are given a cake?''\\[0.5em]
      \hfill{}\emph{Richard Feynman}\\
      \hfill{}\emph{(Feynman Lectures on Physics)}%
    };
  \end{overlay}
\end{frame}

\begin{frame}{\insertsubsection}
  \begin{tikzpicture}
    \tikzset{
      myCircle/.style={
        circle,
        minimum size=27mm,
        white,
        fill=mittelblau,
        align=center,
        inner sep=0mm,
      }
    }
    \node[myCircle] at (0,0) (parameters) {%
      \textbf{Parameters}\\$\*x = (x_1, x_2, \dots)$%
    };
    \node[myCircle] at (3.5,0) (model) {\textbf{Model}};
    \node[myCircle] at (7,0) (simulation) {\textbf{Simulation}};
    \uncover<2->{
      \node[myCircle,fill=hellblau] at (7,-3.5) (reality) {\textbf{Reality}};
    }
    \draw[->,line width=1mm,anthrazit,>=stealth] (parameters) -- (model);
    \draw[->,line width=1mm,anthrazit,>=stealth] (model) -- (simulation);
    \uncover<2->{
      \draw[<->,dots,line width=1mm,anthrazit,>=stealth]
          (simulation) to[bend left=45] (reality);
    }
    \uncover<3->{
      \draw[->,dots,line width=1mm,anthrazit,>=stealth] (reality)
          to[bend left=45] (parameters);
      \path[
        postaction={
          textalongpath={20mm}{%
            |\color{mittelblau}\bfseries|Inverse problem%
          },
          /pgf/decoration/raise=-5mm
        }
      ] (parameters) to[bend right=45] (reality);
    }
    \node[mittelblau] at (3.5,1.6) {\textls[200]{\textbf{Forward problem}}};
    \uncover<2->{\node[anthrazit] at (7.3,-1.75) {\emph{Matches?}};}
  \end{tikzpicture}
  
  \begin{overlay}
    \node[text width=48mm] at (109,12.5) {%
      \heading{Optimization problem:}\\
      Given $\objfun\colon [0, 1]^d \to \real$.\\
      Find $\xopt = \vecargmin_{\*x}\, \objfun(\*x)$.\\
      Example:\\
      $\objfun(\*x) \ceq |\text{reality} - \text{simulation}|^2$\\[1em]
      \heading{Assumptions:}
      \begin{itemize}
        \item
        $\objfun \colon [0, 1]^d \to \real$ black-box
        
        \item
        $\objfun$ computationally expensive
        
        \item
        $d$ moderate ($\le 10$)
        
        \item
        $\objfun$ sufficiently smooth
      \end{itemize}%
    };
  \end{overlay}
\end{frame}



\subsection{Sparse Grid Surrogates by Interpolation}

\begin{frame}{\insertsubsection}
  \begin{overlay}
    \node[text width=42mm] at (50,12) {%
      \heading{%
        \vvisible<1|handout:0>{Objective function}%
        \vvisible<2-3|handout:0>{Full grid}%
        \vvisible<4|handout:0>{Regular sparse grid}%
        %\vvisible<5|handout:0>{Dimensionally adaptive sparse grid}%
        \vvisible<5->{Spatially adaptive sparse grid}%
      }%
    };
    \node<1|handout:0> at (50,28) {$\objfun$};
    \only<2->{
      \node[text width=45mm] at (50,28) {%
        Surrogate $\sgintp \approx \objfun$\alt<1-2>{\\}{\\[0.5em]}%
        \only<1-2|handout:0>{%
          $
            \sgintp
            %\ceq \sum_{\*i=\*0}^{\*2^{\*l}} c_\*i \basis{\*l,\*i}
            \ceq \sum_{i_1=0}^{2^{l_1}} \dotsb \sum_{i_d=0}^{2^{l_d}}
            c_\*i \basis{\*l,\*i}
          $%
        }%
        \only<3|handout:0>{%
          $
            \sgintp
            \ceq \sum_{\norminfty{\*l} \le n} \sum_{\*i \in \hiset{\*l}}
            \alpha_{\*l,\*i} \basis{\*l,\*i}
          $%
        }%
        \only<4|handout:0>{%
          $
            \sgintp
            \ceq \sum_{\normone{\*l} \le n} \sum_{\*i \in \hiset{\*l}}
            \alpha_{\*l,\*i} \basis{\*l,\*i}
          $%
        }%
        %\only<5|handout:0>{%
        %  $
        %    \sgintp
        %    \ceq \sum_{\*l \in L} \sum_{\*i \in \hiset{\*l}}
        %    \alpha_{\*l,\*i} \basis{\*l,\*i}
        %  $%
        %}%
        \only<5->{%
          $
            \sgintp
            \ceq \sum_{(\*l, \*i) \in \liset}
            \alpha_{\*l,\*i} \basis{\*l,\*i}
          $%
        }%
      };
    }
    \node[text width=75mm] at (4,67) {%
      {%
        \uncover<2->{%
          Grid points
          $\gp{\*l,\*i} \ceq (i_1 2^{-l_1},\, \dotsc,\, i_d 2^{-l_d})$%
        }\\
        \uncover<3->{%
          Indices
          $\hiset{\*l} \ceq \hiset{l_1} \times \dotsb \times \hiset{l_d}$
          with $\hiset{0} \ceq \{0, 1\}$ and
          $\hiset{l_t} \ceq \{1, 3, 5, \dotsc, 2^{l_t} - 1\}$ for $l_t > 0$%
        }%
      }%
    };
    \node at (2,12) {%
      \includegraphics<1|handout:0>{interpolant_1}%
      \includegraphics<2-3|handout:0>{interpolant_2}%
      \includegraphics<4|handout:0>{interpolant_3}%
      %\includegraphics<5|handout:0>{interpolant_4}%
      \includegraphics<5->{interpolant_5}%
    };
    %\node at (45,12) {%
    %  \includegraphics<2-3>{sg_3}%
    %  \includegraphics<4>{sg_5}%
    %%  \includegraphics<5>{sg_7}%
    %  \includegraphics<5->{sg_9}%
    %};
    \node at (90,12) {%
      \guncover<2->{\includegraphics<1-2|handout:0>{sg_1}}%
      \includegraphics<3|handout:0>{sg_2}%
      \includegraphics<4|handout:0>{sg_3}%
      %\includegraphics<5|handout:0>{sg_4}%
      \includegraphics<5->{sg_5}%
    };
  \end{overlay}
\end{frame}



\subsection{B-Splines for Sparse Grids}

\begin{frame}{\insertsubsection}
  \begin{overlay}
    \node at (3,10) {%
      \vvisible<1|handout:0>{\includegraphics{hierarchicalBasis_1}}%
      \includegraphics<2->{hierarchicalBasis_3}%
    };
    \node at (55,33) {%
      \includegraphics<1|handout:0>{interpolant_5}%
      \includegraphics<2->{interpolant_6}%
    };
    \node at (62,16) {%
      Cardinal B-spline:\\[0.1em]
      $
        \hphantom{\cardbspl{p}}\mathllap{\cardbspl{0}}
        \ceq \charfun{\hopint{0, 1}}
      $\\[0.3em]
      %$\cardbspl{p}(x) \ceq \int_0^1\!\! \cardbspl{p-1}(x-y) \dy$%
      $\cardbspl{p} \ceq \cardbspl{p-1} \convolution \cardbspl{0}$%
    };
    \node[circle,white,fill=mittelblau,inner sep=1mm] at (90,4) {%
      \alt<1|handout:0>{$p = 1$}{$p = 3$}%
    };
    \uncover<3->{
      \fill[mittelblau] (254,42.92) circle (150mm);
      \node[white,text width=50mm] at (108,6) {%
        \hspace{10mm}\textbf{Advantages:}\\[0.8em]
        \setbeamercolor{itemize item}{fg=white}%
        \begin{itemize}
          \item
          Basis of spline space\\
          \follows Higher order of convergence
          
          \item
          $\Cp{p-1}$-smoothness %\follows Gradient-based optimization
          
          \item
          Special case: Hat functions
          %\follows minimally invasive implementation
          
          \vspace{0.8em}
          
          \item
          Compactly supported %\follows locality
          
          \item
          $p$ as additional degree of freedom %\follows $h$-$p$ adaptivity
          
          \item
          Non-uniform knot sequences possible
          
          \item
          many ``nice'' numerical properties
        \end{itemize}%
      };
    }
  \end{overlay}
\end{frame}



\subsection{Overview}

\begin{frame}{B-Splines for Sparse Grids}{%
  Algorithms and Application to\\
  Higher-Dimensional Optimization%
}
  \begin{overlay}
    \node[text width=80mm] at (15,22) {%
      \setbeamercolor{item}{fg=mittelblau}%
      \tableofcontents[hideallsubsections]%
    };
    \node at (80,0) {\includegraphics[scale=\pngscale]{overview}};
  \end{overlay}
\end{frame}
