\section{Real-World Applications}



\begin{frame}<handout:0>[noframenumbering]
  \begin{overlay}
    \sectionCircle
    \subsectionCircle{50}{60}{Topology\\Optimization}{1}
    \subsectionCircle{110}{60}{Dynamic Portfolio\\Choice Models}{0}
  \end{overlay}
\end{frame}



\subsection{Topology Optimization}

\begin{frame}{\insertsubsection}{Homogenization and Two-Scale Approach}
  \begin{overlay}
    \node[anchor=south] at (13,72) {\includegraphics{microCell_1}};
    \node[anchor=south] at (36,72) {\includegraphics{microCell_3}};
    \node[anchor=south] at (60,72) {\includegraphics{microCell_2}};
    \node[anchor=south] at (85,72) {\includegraphics{microCell_4}};
    \node[anchor=south] at (112,76) {\includegraphics{microCell_13}};
    \node[anchor=south] at (142,76) {\includegraphics{microCell_14}};
    {
      \transparent{\uncoveralpha}
      \fill<1-2|handout:0>[white] (-10,45) rectangle (170,84);
    }
    \node[text width=50mm] at (4,15) {%
      \hspace{15mm}
      $
        \min_{\densglobal\colon \objdomain \to \clint{0, 1}}\,
        \compliance(\densglobal)
      $\\[0.5em]
      with
      $
        \compliance(\densglobal)
        \ceq \int_{\objdomain} \tr{\force}
        \displacement_{\densglobal}(\tilde{\*x}) \diff\tilde{\*x}
      $\\[0.5em]
      s.t.
      $
        \frac{
          \int_{\objdomain} \densglobal(\tilde{\*x}) \diff\tilde{\*x}
        }{
          \vol{\objdomain}
        }
        \le \densub
      $%
    };
    \node<2-> at (54.4,12.2) {\includegraphics{twoScale_1}};
    \node<1|handout:0> at (53,16) {\includegraphics{topoOptExample_1}};
    {
      \transparent{0.15}
      \node<2-> at (53,16) {\includegraphics{topoOptExample_2}};
    }
    \uncover<3->{
      \node[anchor=north,align=center] at (13,72.5) {%
        Cross\vphantom{hd}\\[-0.3em]
        ($d = 2$)%
      };
      \node[anchor=north,align=center] at (36,72.5) {%
        Framed cross\vphantom{hd}\\[-0.3em]
        ($d = 4$)%
      };
      \node[anchor=north,align=center] at (62,72.5) {%
        Sheared cross\vphantom{hd}\\[-0.3em]
        ($d = 3$)%
      };
      \node[anchor=north,align=center] at (87,72.5) {%
        SF cross\vphantom{hd}\\[-0.3em]
        ($d = 5$)%
      };
      \node[anchor=north,align=center] at (112,76.5) {%
        3D cross\vphantom{hd}\\[-0.3em]
        ($d = 3$)%
      };
      \node[anchor=north,align=center] at (140,76.5) {%
        3D sheared cross\vphantom{hd}\\[-0.3em]
        ($d = 5$)%
      };
    }
  \end{overlay}
\end{frame}



\begin{frame}{\insertsubsection}{Cholesky Factor Interpolation}
  \begin{overlay}
    \tikzset{
      myCircle/.style={
        circle,
        white,
        fill=mittelblau,
        anchor=west,
        inner sep=1mm,
        minimum size=10mm,
      },
    }
    
    \fill[anthrazit!20,rounded corners=3mm] (-10,18) rectangle (92,44);
    \node at (3.5,20) {%
      \heading{Offline phase (generation of interpolant):}%
    };
    \node[anchor=center] (offline1) at (13,34) {%
      \includegraphics{topoOptOfflineOnline_1}%
    };
    \node[anchor=center] (offline2) at (38,34) {%
      \includegraphics{topoOptOfflineOnline_2}%
    };
    \node<4->[anchor=center] (offline3) at (63,34) {%
      \includegraphics{topoOptOfflineOnline_3}%
    };
    \node[myCircle] (offline4) at (79,34) {%
      \alt<1-3|handout:0>{$\etensorintp$}{$\cholfactorintp$}%
    };
    \draw[->] (offline1) -- (offline2);
    \alt<1-3|handout:0>{
      \draw[->] (offline2) -- (offline4);
    }{
      \draw[->] (offline2) -- (offline3);
      \draw[->] (offline3) -- (offline4);
    }
    
    \fill[anthrazit!20,rounded corners=3mm] (-10,48) rectangle (92,76);
    \node at (3.5,50) {%
      \heading{Online phase (one iteration of optimizer):}%
    };
    \node[anchor=center] (online1) at (13,65) {%
      \includegraphics{topoOptOfflineOnline_4}%
    };
    \node<4->[anchor=center] (online2) at (38,65) {%
      \includegraphics{topoOptOfflineOnline_6}%
    };
    \node[anchor=center] (online3) at (63,65) {%
      \includegraphics<1-3|handout:0>{topoOptOfflineOnline_5}%
      \includegraphics<4->{topoOptOfflineOnline_7}%
    };
    \node[myCircle] (online4) at (79,65) {$\complianceintp$};
    \alt<1-3|handout:0>{
      \draw[->] (online1) -- (online3);
    }{
      \draw[->] (online1) -- (online2);
      \draw[->] (online2) -- (online3);
    }
    \draw[->] (online3) -- (online4);
    
    \uncover<2->{
      \node[anchor=north] at (124,2) {%
        \small%
        Minimal eigenvalue of
        \alt<1-3>{$\etensorintp$}{$\etensorcholintp$}%
        \vphantom{$\etensorcholintp$}%
      };
    }
    \node at (94,6) {%
      \guncover<2->{\includegraphics<1-3|handout:0>{cholesky_1}}%
      \includegraphics<4->{cholesky_2}\;\;%
      \guncover<2->{\raisebox{2mm}{\includegraphics{cholesky_3}}}%
    };
    \uncover<3->{
      \node at (100,58) {%
        \heading{Cholesky factorization:}\\
        %$
        %  \etensorcholintp(\*x)
        %  \ceq \tr{\cholfactorintp(\*x)} \cholfactorintp(\*x)
        %$%
        $\etensor(\*x) = \tr{\cholfactor(\*x)} \cholfactor(\*x)$%
      };
      \node at (100,70) {%
        %$
        %  \partialderiv{\partialdiff{} x_t}{\etensorcholintp}(\*x)
        %  = \tr{\cholfactorintp(\*x)} \cdot
        %  \partialderiv{\partialdiff{} x_t}{\cholfactorintp}(\*x) +
        %  \tr{\partialderiv{\partialdiff{} x_t}{\cholfactorintp}(\*x)} \cdot
        %  \cholfactorintp(\*x)
        %$%
        $
          \partialderiv{\partialdiff{} x_t}{\etensor}
          = \tr{\cholfactor} \cdot
          \partialderiv{\partialdiff{} x_t}{\cholfactor} +
          \tr{\partialderiv{\partialdiff{} x_t}{\cholfactor}} \cdot
          \cholfactor
        $%
      };
    }
    
    %\only<5->{
    %  \thesisCircle{52.5mm}{-42.912mm}{35mm}{60mm}{70mm}{49mm}{167}
    %  \thesisCircleCaption{52.5mm}{-42.912mm}{35mm}{127}{53}{%
    %    Error without Cholesky%
    %  }
    %  \thesisCircle{107.5mm}{-42.912mm}{35mm}{137mm}{70mm}{49mm}{167}
    %  \thesisCircleCaption{107.5mm}{-42.912mm}{35mm}{122}{58}{%
    %    Error with Cholesky%
    %  }
    %}
  \end{overlay}
  
  \cite[in prepar.]{Valentin19Gradient}
\end{frame}



\begin{frame}{\insertsubsection}{Numerical Results}
  \begin{overlay}
    %\node at (57,3.5) {\heading{\footnotesize{}Numerical Results}};
    
    \node[anchor=north,text width=60mm,align=center] at (44,9) {%
      Spectral error\\%
      {%
        \small%
        $
          \norm[2]{
            \etensor(\*x) -
            \alt<1>{\etensorintp}{\etensorcholintp}(\*x)
            \vphantom{\etensorcholintp}
          }
        $%
      }%
    };
    \node at (23,21) {\includegraphics{topoOptInterpolationPointwise_3}};
    \node at (18,31) {%
      \includegraphics<1|handout:0>{topoOptInterpolationPointwise_1}%
      \includegraphics<2->{topoOptInterpolationPointwise_2}%
    };
    
    \guncover<3->{
      \node[anchor=north,text width=60mm,align=center] at (110,9) {%
        Relative spectral error\\
        {%
          \small%
          $
            \error^{\chol,\sparse} :=
            \frac{
              \normLtwoscaled{
                \vphantom{\big(}
                \norm[2]{\etensor({\cdot}) - \etensorcholintp({\cdot})}
              }
            }{
              \normLtwoscaled{
                \vphantom{\big(}
                \norm[2]{\etensor({\cdot})}
              }
            }
          $%
        }%
      };
      \node at (74,27) {\includegraphics{topoOptInterpolation_1}};
    }
    
    %\guncover<4->{
    %  \node[anchor=north,text width=60mm,align=center] at (128,9) {%
    %    Optimality-interpolation gap%
    %  };
    %  \node at (102,27) {\includegraphics{topoOptOptimalityGap_1}};
    %}
    %\node at (112,15) {\includegraphics{topoOptOptimalityGap_3}};
    %{
    %  \transparent{\uncoveralpha}
    %  \fill<1-3|handout:0>[white] (111,14) rectangle (170,26);
    %}
  \end{overlay}
  
  \cite[in prepar.]{Valentin19Gradient}
\end{frame}



\begin{frame}{\insertsubsection}{%
  Optimal Structures and Compliance Values (Lower is Better)%
}
  \begin{overlay}
    \node<1-2|handout:0> at (4,17) {%
      \includegraphics<1>{topoOptStructure2D_1}%
      %\includegraphics<2>{topoOptStructure2D_3}%
      \includegraphics<2>{topoOptStructure2D_5}%
      %\includegraphics<4>{topoOptStructure2D_7}%
    };
    \node<1-2|handout:0>[text width=70mm] at (6,58) {%
      {%
        \only<2->{\includegraphics{microCell_5}}%
        \only<1>{\includegraphics{microCell_9}}\hspace{0.41em}%
        \alert<1>{\raisebox{0.7em}{ 74.974}}\qquad%
        \only<1,3->{\includegraphics{microCell_6}}%
        \only<2>{\includegraphics{microCell_10}}%
        \alert<2>{\raisebox{0.7em}{ 67.809}}\\[0.2em]%
        \includegraphics{microCell_7}\hspace{0.41em}%
        \raisebox{0.7em}{ 70.816}\qquad%
        \includegraphics{microCell_8}%
        \raisebox{0.7em}{ 68.602}%
      }%
    };
    
    \node<1-2|handout:0> at (90,15) {%
      \includegraphics<1>{topoOptStructure2D_2}%
      %\includegraphics<2>{topoOptStructure2D_4}%
      \includegraphics<2>{topoOptStructure2D_6}%
      %\includegraphics<4>{topoOptStructure2D_8}%
    };
    \node<1-2|handout:0>[text width=30mm] at (125,10) {%
      {%
        \hspace{0.27em}%
        \only<2->{\includegraphics{microCell_5}}%
        \only<1>{\includegraphics{microCell_9}}\hspace{0.41em}%
        \alert<1>{\raisebox{0.7em}{ 183.68}}\\[0.2em]%
        \only<1,3->{\includegraphics{microCell_6}}%
        \only<2>{\includegraphics{microCell_10}}%
        \alert<2>{\raisebox{0.7em}{ 169.60}}\\[0.2em]%
        \hspace{0.27em}%
        \includegraphics{microCell_7}\hspace{0.41em}%
        \raisebox{0.7em}{ 177.51}\\[0.2em]%
        \includegraphics{microCell_8}%
        \alert<4>{\raisebox{0.7em}{ 174.55}}%
      }%
    };
    
    \node<3->[anchor=center] at (32,47) {%
      \includegraphics<3|handout:0>[scale=0.16]{topoOptStructure658}%
      \includegraphics<4->[scale=0.16]{topoOptStructure660}%
    };
    \node<3->[text width=30mm] at (45,15) {%
      {%
        \only<1-2,4->{\includegraphics{microCell_15}}%
        \only<3|handout:0>{\includegraphics{microCell_17}}\hspace{0.25em}%
        \alert<3|handout:0>{\raisebox{1.5em}{ 247.60}}\\[0.0em]%
        \only<1-3|handout:0>{\includegraphics{microCell_16}}%
        \only<4>{\includegraphics{microCell_18}}%
        \alert<4>{\raisebox{1.5em}{\,162.59}}%
      }%
    };
    
    \node<3->[anchor=center] at (115,52) {%
      \includegraphics<3|handout:0>[scale=0.16]{topoOptStructure659}%
      \includegraphics<4->[scale=0.16]{topoOptStructure661}%
    };
    \node<3->[text width=30mm] at (130,2) {%
      {%
        \only<1-2,4->{\includegraphics{microCell_15}}%
        \only<3|handout:0>{\includegraphics{microCell_17}}\hspace{0.25em}%
        \alert<3|handout:0>{\raisebox{1.5em}{ 169.27}}\\[0.0em]%
        \only<1-3>{\includegraphics{microCell_16}}%
        \only<4>{\includegraphics{microCell_18}}%
        \alert<4>{\raisebox{1.5em}{\,46.171}}%
      }%
    };
  \end{overlay}
  
  \cite[in prepar.]{Valentin19Gradient}
\end{frame}



\begin{frame}<handout:0>[noframenumbering]
  \begin{overlay}
    \sectionCircle
    \subsectionCircle{50}{60}{Topology\\Optimization}{0}
    \subsectionCircle{110}{60}{Dynamic Portfolio\\Choice Models}{1}
  \end{overlay}
\end{frame}



\subsection{Dynamic Portfolio Choice Models}

\begin{frame}{\insertsection}{Introduction}
  \begin{overlay}
    \node[text width=120mm] at (2,13) {%
      \begin{itemize}
        \item
        How should individuals invest their money over their life span?
        
        \item
        Consume all money that is not invested (\follows consumption $c_t$)
        
        \item
        Goal: Maximize total discounted utility!
      \end{itemize}%
    };
    
    \node at (10,36) {%
      \only<1-2,5->{\includegraphics{dynamicPortfolioChoice_1}}%
      \only<3>{\includegraphics{dynamicPortfolioChoice_2}}%
      \only<4>{\includegraphics{dynamicPortfolioChoice_3}}%
    };
    
    \uncover<2->{
      \node[text width=52mm] at (99,30) {%
        \begin{equation*}
          \max \sum_{t=0}^T \riskav^t
           \expectation[t]{
            \utilityfun(
              \consume_t(
                \alert<3>{\state_t},
                \alert<4>{\policy_t},
                \alert<5>{\stochastic_t}
              )
            )
          }
        \end{equation*}
        
        \vspace*{-0.5em}
        
        \begin{equation*}
          u(c_t)
          = \frac{c_t^{1 - \riskav}}{1 - \riskav}
          = \raisebox{-0.5\height+0.4em}{
            \tikz{
              \begin{axis}[
                domain=1:4,
                width=23mm,
                height=23mm,
                ticks=none,
                axis lines=left
              ]
                \addplot[smooth,samples=200,black]{x^(1-3.5)/(1-3.5)};
              \end{axis}
            }
          }
        \end{equation*}
        
        \vspace*{-0.3em}
        
        \begin{itemize}
          \itemsep=0em
          \item
          State variables \alert<3>{$\state_t$}
          %Cannot be directly influenced (wealth, income, etc.)
          
          \item
          Policy variables \alert<4>{$\policy_t$}
          %Investment decisions in time $t$
          
          \item
          Stochastic variables \alert<5>{$\stochastic_t$}
          %Random influence on stock market
          %Patience $\patience < 1$ ($\patience = 0.97$),\\
          %risk aversion $\riskav > 1$ ($\riskav = 3.5$)
        \end{itemize}
      };
    }
  \end{overlay}
\end{frame}



{%
  \newcommand*{\getscale}{
    \pgfgettransformentries{%
      \myxscaletmp%
    }{\@tempa}{\@tempa}{\myyscaletmp}{\@tempa}{\@tempa}
    \gdef\myxscale{\myxscaletmp}
  }%
  \newcommand*{\drawcircle}[2][]{
    \node[
      circle,minimum size=60mm,fill=mittelblau!#2,draw=mittelblau,
    ] at (0mm,0mm) (myCircle#2) {};
    \IfStrEq{#1}{noArrow}{}{
      \getscale
      \pgfmathparse{150+20/\myxscale^0.8}
      \let\myin\pgfmathresult
      \draw[->] (myCircle#2) to[
        out=150,in=\myin,looseness=2.3,
      ] node[below,scale=1/\myxscale] {\mytext} (myCircle#2);
    }
  }%
  \newcommand*{\drawtext}[1]{
    \getscale
    \centerarc[
      draw=none,postaction={decorate},decoration={
        text along path,
        text={#1},
        text align=center
      },
    ](0mm,0mm)(180:0:30mm-1em/\myxscale);
  }%
  \newenvironment*{circlescope}{
    \getscale
    %\begin{scope}[scale=0.8,shift={(5mm,-5mm)},transform shape]
    \begin{scope}[
      scale=1-0.15/\myxscale,
      shift={(3mm/\myxscale,-3mm/\myxscale)},
      transform shape,
    ]
  }{
    \end{scope}
  }%
  \newsavebox{\mybox}%
  \sbox{\mybox}{%
    \begin{tikzpicture}
      \draw[dashed] (0mm,9mm) -- (84mm,9mm);
      \draw[dashed] (-4.3mm,-8mm) -- (71.2mm,-48.2mm);
      \begin{scope}[rotate=-130]
        \node[
          circle,
          minimum size=18mm,
          fill=mittelblau!30,
          draw=mittelblau,
          inner sep=0mm,
        ] at (0mm,0mm) (optimize) {\texttt{optimize}};
        \begin{scope}[
          shift={(40mm/2,40mm*sin(60))},
          transform shape,
        ]
          \begin{scope}[
            rotate=130,
            scale=18mm/60mm+0.15,
            transform shape,
          ]
            \getscale
            \drawcircle[noArrow]{30}
            \drawtext{|\ttfamily|refine}
            \begin{scope}[
              scale=1-0.15/\myxscale,
              shift={(3mm/\myxscale,-3mm/\myxscale)},
              transform shape,
            ]
              \def\mytext{}
              \drawcircle{50}
              \getscale
              \node[scale=1/\myxscale] at (0mm,0mm) {\texttt{optimize}};
            \end{scope}
          \end{scope}
        \end{scope}
        \node[
          circle,
          minimum size=18mm,
          fill=mittelblau!30,
          draw=mittelblau,
          inner sep=0mm,
          text width=15mm,
          align=center,
        ] at (40mm,0mm) (interpolate) {\ttfamily{}inter-\\polate};
        \draw[->] (optimize) to[
          bend left=30,
        ] (myCircle30);
        \draw[->] (myCircle30) to[
          bend left=30,
        ] (interpolate);
        \draw[->] (interpolate) to[
          bend left=30,
        ] node[above,rotate=50] {$t \gets (t - 1)$} (optimize);
      \end{scope}
      \begin{scope}[shift={(84mm,-21mm)}]
        \drawcircle[noArrow]{20}
        \drawtext{|\ttfamily|optimize}
        \begin{circlescope}
          \def\mytext{\hspace*{5mm}$\state_t$}
          \drawcircle{30}
          \drawtext{|\ttfamily|optimizeSinglePoint}
          \begin{circlescope}
            \def\mytext{\hspace*{5mm}$\policy_t$}
            \drawcircle{40}
            \drawtext{|\ttfamily|evalObjfunGrad}
            \begin{circlescope}
              \def\mytext{\hspace*{5mm}$\stochastic_t$}
              \drawcircle{50}
              \drawtext{|\ttfamily|evalQuadPoint}
              \begin{circlescope}
                \def\mytext{\hspace*{4mm}$o$}
                \drawcircle{60}
                \getscale
                \node[
                  text width=30mm,align=center,scale=1/\myxscale,
                ] at (0mm,0mm) {%
                  \ttfamily{}evalInterp-\\PartDeriv%
                };
              \end{circlescope}
            \end{circlescope}
          \end{circlescope}
        \end{circlescope}
      \end{scope}
    \end{tikzpicture}%
  }
  
  \newcommand*{\mystate}{
    \only<1>{\state_t^{\phantom{(k)}}}
    \only<2->{\alert<2>{\state_t^{(k)}}}
  }%
  \newcommand*{\myvaluefun}[1]{
    \only<1-2>{\valuefun_{#1}^{\phantom{\sparse}}}
    \only<3->{\alert<3>{\valueintp_{#1}}}
  }%
  \begin{frame}{\insertsubsection}{Bellman Equation}
    \begin{overlay}
      \node[anchor=north] at (80,16) {%
        $
          \myvaluefun{t}(\mystate)
          = \alert<4>{\max}_{\policy_t}
          \;\biggl(\!
            \utilityfun(\consume_t(\mystate, \policy_t)) +
            \patience \expectationsign[t]\bigl[
              \myvaluefun{t+1}(
                \;\underbrace{
                  \statefun_t(\mystate, \policy_t, \stochastic_t)
                }_{
                  \mathclap{\text{$\state_{t+1}$ (state transition)}}%
                }\;
              )
            \bigr]
          \biggr)
        $%
      };
      
      \node[anchor=north] at (80,35) {\scalebox{0.6}{\usebox{\mybox}}};
      
      \node<5>[
        circle,
        white,
        fill=mittelblau,
        anchor=center,
        text width=42mm,
      ] at (80,42) {%
        \vspace{-1em}%
        \begin{itemize}
          \setbeamercolor{itemize item}{fg=white}%
          \item
          Interpolation
          
          \item
          Spatial adaptivity
          
          \item
          Gradient-based opt.
          
          \item
          CE transformation
          
          \item
          SG quadrature
          
          \item
          Extrapolation schemes
          
          \item
          State space cropping
          
          \item
          Euler equation errors
          
          \item
          Monte Carlo sim.
        \end{itemize}%
      };
    \end{overlay}
    
    \cite[in preparation]{Pflueger19Solving}
  \end{frame}%
}



\begin{frame}{\insertsubsection}{Interpolation and Euler Error}
  \begin{overlay}
    \fill[mittelblau,rounded corners=3mm] (101,-10) rectangle (170,38);
    
    \node[white,text width=57mm] at (103,2) {%
      \textbf{Transaction costs problem:}\vspace{-0.2em}
      \begin{itemize}
        \setbeamercolor{itemize item}{fg=white}
        \item
        $d$ state variables\\
        (stock fractions $\stock_{t,j}$)
        
        \item
        $2d+1$ policy variables\\
        (bonds $\normbond_t$, buys $\normbuy_{t,j}$,
        sells $\normsell_{t,j}$)
        
        \item
        $d$ stochastic variables\\
        (stock return rates $\stockreturn_{t,j}$)
      \end{itemize}%
    };
    
    %\node<1|handout:0> at (4,13) {%
    %  \includegraphics{financeSolution2D_1}\;%
    %  \includegraphics{financeSolution2D_2}\;%
    %  \includegraphics{financeSolution2D_4}\\
    %  \includegraphics{financeSolution2D_6}\;%
    %  \includegraphics{financeSolution2D_3}\;%
    %  \includegraphics{financeSolution2D_5}%
    %};
    
    \node<1|handout:0> at (4,13) {%
      \includegraphics{financeSolution2D_7}\,%
      \includegraphics{financeSolution2D_8}\,%
      \includegraphics{financeSolution2D_10}\\
      \includegraphics{financeSolution2D_12}\,%
      \includegraphics{financeSolution2D_9}\,%
      \includegraphics{financeSolution2D_11}%
    };
  
    \node<2|handout:0> at (4,13) {%
      \includegraphics{financeSolution2D_13}\,%
      \includegraphics{financeSolution2D_14}\,%
      \includegraphics{financeSolution2D_16}\\
      \includegraphics{financeSolution2D_18}\,%
      \includegraphics{financeSolution2D_15}\,%
      \includegraphics{financeSolution2D_17}%
    };
    
    \node<3-> at (4,13) {%
      \includegraphics{financeSolution2D_19}\,%
      \includegraphics{financeSolution2D_20}\,%
      \includegraphics{financeSolution2D_22}\\
      \includegraphics{financeSolution2D_24}\,%
      \includegraphics{financeSolution2D_21}\,%
      \includegraphics{financeSolution2D_23}%
    };
    
    \node at (103,39) {%
      \includegraphics<1|handout:0>{financePointwiseError_1}%
      \includegraphics<2|handout:0>{financePointwiseError_2}%
      \includegraphics<3->{financePointwiseError_3}%
      \quad\raisebox{3.50mm}{\includegraphics{financePointwiseError_4}}%
    };
    \node[text width=30mm] at (122,40) {%
      $\weightedeulererror_t(\vstock_t)$\\
      ($
        \ngp_t =
        \only<1|handout:0>{129}%
        \only<2|handout:0>{889}%
        \only<3->{5159}
      $)%
    };
  \end{overlay}
  
  \cite[in preparation]{Pflueger19Solving}
\end{frame}



\begin{frame}{\insertsubsection}{Monte Carlo Simulation}
  \begin{overlay}
    \node at (4,20) {%
      \includegraphics{financeSimulation_1}%
    };
    \node at (55,20) {%
      \includegraphics{financeSimulation_2}%
    };
    \node at (106,20) {%
      \includegraphics{financeSimulation_3}%
    };
    \node[circle,white,fill=mittelblau,inner sep=1mm] at (36,25) {%
      $d = 3$\vphantom{$d = 345$}%
    };
    \node[circle,white,fill=mittelblau,inner sep=1mm] at (87,25) {%
      $d = 4$\vphantom{$d = 345$}%
    };
    \node[circle,white,fill=mittelblau,inner sep=1mm] at (138,25) {%
      $d = 5$\vphantom{$d = 345$}%
    };
  \end{overlay}
  
  \cite[in preparation]{Pflueger19Solving}
\end{frame}



\subsection{Other Applications from My Thesis}

\begin{frame}{\insertsubsection}
  \begin{overlay}
    \only<1|handout:0>{
      \thesisCircle{59mm}{-45mm}{35mm}{112mm}{62mm}{30mm}{176}
      \thesisCircleCaption{59mm}{-45mm}{35mm}{127}{53}{%
        Musculoskeletal model%
      }
      \thesisCircle{112.5mm}{-25mm}{22mm}{100mm}{65mm}{40mm}{184}
      \thesisCircleCaption{112.5mm}{-25mm}{22mm}{158}{22}{%
        Equilibrium elbow angle%
      }
      \thesisCircle{108mm}{-65mm}{18mm}{143mm}{75mm}{52mm}{220}
      \thesisCircleCaption{108mm}{-65mm}{18mm}{135}{45}{%
        Future work%
      }
    }
    \only<2|handout:0>{
      \thesisCircle{49mm}{-45mm}{35mm}{110mm}{150mm}{80mm}{132}
      \thesisCircleCaption{49mm}{-45mm}{35mm}{115}{65}{%
        Test problems%
      }
      %\thesisCircle{80mm}{-45mm}{35mm}{110mm}{70mm}{49mm}{133}
      %\thesisCircleCaption{80mm}{-45mm}{35mm}{127}{53}{%
      %  Constrained test functions%
      %}
      \thesisCircle{114mm}{-48mm}{30mm}{105mm}{80mm}{70mm}{141}
      \thesisCircleCaption{114mm}{-48mm}{30mm}{122}{58}{%
        Optimality gaps%
      }
    }
    \only<3>{
      \thesisCircle{44mm}{-48mm}{30mm}{110mm}{85mm}{60mm}{146}
      \thesisCircleCaption{44mm}{-48mm}{30mm}{138}{42}{%
        Fuzzy extension principle%
      }
      \thesisCircle{108.5mm}{-39mm}{35mm}{100mm}{185mm}{60mm}{149}
      \thesisCircleCaption{108.5mm}{-39mm}{35mm}{145}{35}{%
        Fuzzy Ritter{--}Novak grid generation%
      }
    }
  \end{overlay}
\end{frame}
